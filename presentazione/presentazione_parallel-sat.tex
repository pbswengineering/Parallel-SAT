\documentclass[12pt,xcolor=dvipsnames]{beamer}

\usetheme{Dresden}
% Pittsburgh: nice and empty
% Marburg: stylish right side bar

\usepackage[utf8]{inputenc}
\usepackage[italian]{babel}
\usepackage{listings}

\frenchspacing

\lstset {
	basicstyle=\footnotesize,       % the size of the fonts that are used for the code
	numberstyle=\footnotesize,      % the size of the fonts that are used for the line-numbers
	stepnumber=2,                   % the step between two line-numbers. If it's 1 each line will be numbered
	numbersep=5pt,                  % how far the line-numbers are from the code
	backgroundcolor=\color{white},  % choose the background color. You must add \usepackage{color}
	showspaces=false,               % show spaces adding particular underscores
	showstringspaces=false,         % underline spaces within strings
	showtabs=false,                 % show tabs within strings adding particular underscores
	frame=single,                   % adds a frame around the code
	tabsize=4,                      % sets default tabsize to 2 spaces
	captionpos=b,                   % sets the caption-position to bottom
	breaklines=true,                % sets automatic line breaking
	breakatwhitespace=false,        % sets if automatic breaks should only happen at whitespace
	escapeinside={\%*}{*)}          % if you want to add a comment within your code
}

\setbeamercovered{transparent}
\setbeamertemplate{background canvas}{\includegraphics[width=\paperwidth,height=\paperheight]{unipg-back.png}}
\usecolortheme[named=Maroon]{structure}

% 25mm is very good with themes that allows for more space
\pgfdeclareimage[height=15mm]{logo}{unipg.png}
\pgfdeclareimage[height=35mm]{albero}{albero.png}
\pgfdeclareimage[height=50mm]{depth-first}{depthfirst.png}
\pgfdeclareimage[height=65mm]{grafico-tempi}{grafico-tempi.png}
\pgfdeclareimage[height=65mm]{grafico-speed-up}{grafico-speed-up.png}
\pgfdeclareimage[height=65mm]{grafico-efficienza}{grafico-efficienza.png}

\title{Implementazione di un SAT solver parallelo basato sull'algoritmo DPLL}
\titlegraphic{\pgfuseimage{logo}}
\author{Paolo Bernardi}
\subtitle{Esame di Applicazioni e Calcolo in Rete, corso avanzato}
\institute{Università degli Studi di Perugia - Corso di Laurea in Informatica}
\date{30 settembre 2008}

\begin{document}

%\transduration{1}

\begin{frame}
	\titlepage
\end{frame}

\part{La teoria}

\begin{frame}
	\partpage
\end{frame}

\section{Il problema SAT}

\begin{frame}
	\frametitle{Il problema SAT}
	\begin{itemize}
	 \item SAT: Soddisfacibilità booleana/proposizionale \pause
	 \item 1971, Stephen Cook: SAT è NP-completo \pause
	 \item Qualsiasi problema NP-completo può essere ricondotto a SAT in tempo polinomiale
	\end{itemize}
\end{frame}

\begin{frame}
	\frametitle{Conjunctive Normal Form}
	\begin{itemize}
	 \item I SAT solver lavorano su formule in CNF \pause
	 \item Qualsiasi formula booleana può essere ricondotta in CNF in tempo polinomiale \pause
     \item Formula in CNF: Clausola1 and Clausola2 and \ldots{} \pause
     \item Clausola: Letterale1 or Letterale2 or \ldots{} \pause
     \item Letterale: Variabile1/not Variabile1
	\end{itemize}
\end{frame}

\begin{frame}
	\frametitle{Le applicazioni di SAT}
	\begin{itemize}
	 \item Verifica di microprocessori \pause
	 \item Model checking \pause
     \item Ragionamento automatico \pause
     \item \ldots{}
	\end{itemize}
\end{frame}

\section{Il backtracking}

\begin{frame}
	\frametitle{Il backtracking}
    \begin{center}
    \pgfuseimage{albero}
    \end{center}
	\begin{itemize}
	 \item Visita in depth-first dell'albero delle valutazioni \pause
	 \item $n$ variabili $\rightarrow$ albero da $2^{n}-1$ nodi \pause
     \item Complessità esponenziale
	\end{itemize}
\end{frame}

\begin{frame}
	\frametitle{Il backtracking}
    \begin{center}
    \pgfuseimage{depth-first}
    \end{center}
\end{frame}

\section{La procedura DPLL}

\begin{frame}
	\frametitle{La procedura DPLL}
	\begin{itemize}
	 \item 1962: Davis, Putnam, Logemann, Loveland \pause
	 \item Backtracking specifico per SAT \pause
     \item Se può, assegna un valore fisso ad una variabile \pause
     \item Se può, determina la correttezza di una soluzione parziale
	\end{itemize}
\end{frame}

\begin{frame}
	\frametitle{Assegnare un valore fisso}
	\textbf{Asserzione:} clausole di una sola variabile: TRUE se è positiva, FALSE se è negata \pause

    \begin{center}
        \ldots{} and $variabile$ and \ldots{} $\rightarrow$ $variabile = T$ \pause

        \ldots{} and $($not $variabile)$ and \ldots{} $\rightarrow$ $variabile = F$
    \end{center}
\end{frame}

\begin{frame}
	\frametitle{Assegnare un valore fisso}
	\textbf{Letterale puro:} variabile sempre positiva o negata in tutta la formula: rispettivamente TRUE o FALSE \pause

    \begin{center}
        $(x$ or $y$ or $($not $z))$ and $(x$ or $($not $y)$ or $z)$ $\rightarrow$ $x = T$ \pause

        $(($not $x)$ or $y)$ and $(($not $x)$ or $($not $y))$ $\rightarrow$ $x = F$
    \end{center}
\end{frame}

\begin{frame}
	\frametitle{Correttezza di soluzioni parziali}
	\begin{itemize}
     \item Consente di evitare di arrivare alle foglie dell'albero \pause
	 \item La formula viene modificata: ogni nodo dell'albero deve avere $(Valutazione, Formula)$
	\end{itemize}
\end{frame}

\begin{frame}
	\frametitle{Correttezza di soluzioni parziali}
	\begin{itemize}
	 \item \textbf{Sussunzione:} elimina \textit{clausole} con variabili positive (negate) con valore TRUE (FALSE) \pause
     \item Formula priva di clausole $\rightarrow$ soluzione
	\end{itemize} \pause
    \begin{center}
        \ldots{} and $(x$ or $y$ or $z)$ and \ldots{}, $x = T$ $\rightarrow$ sussunzione \pause

        \ldots{} and $(($not $x)$ or $y)$ and \ldots{}, $x = F$ $\rightarrow$ sussunzione
    \end{center}
\end{frame}

\begin{frame}
	\frametitle{Correttezza di soluzioni parziali}
	\begin{itemize}
     \item \textbf{Risoluzione unitaria:} elimina {variabili} positive (negate) con valore FALSE (TRUE) \pause
     \item Formula con clausola senza variabili $\rightarrow$ soluzione errata
	\end{itemize} \pause
    \begin{center}
        \ldots{} and $(x$ or $y)$ and \ldots{}, $x = F$ $\rightarrow$ \ldots{} and $y$ and \ldots{} \pause

        \ldots{} and $(($not $x)$ or $y)$ and \ldots{}, $x = T$ $\rightarrow$ \ldots{} and $y$ and \ldots{}
    \end{center}
\end{frame}

\begin{frame}
	\frametitle{Ritorno al backtracking}
	\begin{itemize}
	 \item Quando non posso applicare le regole precedenti: \textbf{sdoppiamento} \pause
	 \item Origine dell'esponenzialità \pause
     \item Quale variabile scegliere?
	\end{itemize}
\end{frame}

\section{Parallelizzare la DPLL}

\begin{frame}
	\frametitle{Parallelizzare la DPLL}
	\begin{itemize}
	 \item Operazioni su un singolo nodo veloci? \pause
     \item \ldots{} niente parallelismo a grana fine \pause
	 \item Numero di nodi esponenziale? \pause
     \item \ldots{} sottoalberi processati in parallelo (grana grossa) \pause
     \item Impossibile valutare il tempo per un sottoalbero? \pause
     \item \ldots{} dynamic task-farm
	\end{itemize}
\end{frame}

\part{Implementazione}

\begin{frame}
	\partpage
\end{frame}

\section{Strumenti utilizzati}

\begin{frame}[fragile]
	\frametitle{Alla base di tutto}
	\begin{itemize}
	 \item C++: programmazione a oggetti e STL \pause
	 \item MPI 2: interfaccia per il C++
	\end{itemize} \pause
\begin{lstlisting}[language=c++]
// Versione C
int ok = MPI_Iprobe(MPI_ANY_SOURCE, MPI_ANY_TAG, MPI_COMM_WORLD, &status);

// Versione C++
bool ok = MPI::COMM_WORLD.Iprobe(MPI::ANY_SOURCE, MPI::ANY_TAG);
\end{lstlisting}
\end{frame}

\begin{frame}
	\frametitle{Per lo sviluppo}
	\begin{itemize}
	 \item Cluster di 4 macchine virtuali XEN \pause
	 \item Processore dual-core, 96Mb di RAM \pause
	 \item GCC 4 \pause
     \item OpenMPI
	\end{itemize}
\end{frame}

\begin{frame}
	\frametitle{Per il testing}
	\begin{itemize}
	 \item Chemgrid: 9 processori dual-core, 1Gb di RAM\pause
	 \item Intel C++ Compiler 10.1 \pause
	 \item MPICH2
	\end{itemize}
\end{frame}

\section{Formato del file di input}

\begin{frame}
	\frametitle{File in formato CNF}
	\begin{itemize}
	 \item Standard de facto per i SAT solver \pause
     \item File di testo ``row-oriented'' \pause
	 \item Commenti che iniziano con ``c'' \pause
     \item Intestazione: ``p cnf num\_clausole num\_variabili'' \pause
	 \item Variabili indicizzate e indicate dal loro indice \pause
     \item Variabili negate $\rightarrow$ numeri negativi \pause
     \item Clausole $\rightarrow$ numeri separati da spazi e terminati con 0
	\end{itemize}
\end{frame}

\begin{frame}[fragile]
    \frametitle{Esempio di file CNF}
    \begin{itemize}
     \item Codifica della formula \verb|(a or b or c) and (d or (not a)) and (not b)|
    \end{itemize} \pause
\begin{lstlisting}
c File di esempio
p cnf 3 4
1 2 3 0
4 -1 0
-2 0
c Fine dell'esempio
\end{lstlisting}
\end{frame}

\section{Struttura del programma}

\begin{frame}
    \frametitle{L'uso di MPI}
    \begin{itemize}
     \item Chiamate MPI incapsulate nei metodi della classe Communications \pause
     \item Oggetti complessi serializzati prima dell'invio \pause
     \item Primitive utilizzate: Send, Recv, Iprobe, Bcast, Wtime
    \end{itemize}
\end{frame}

\begin{frame}[fragile]
    \frametitle{Main: master}
\begin{lstlisting}[language=c++]
Node n(argv[1]);
double start_time = MPI::Wtime();

if (dpll(n, c))
    cout << "SAT" << endl;
else
    cout << "UNSAT" << endl;

double end_time = MPI::Wtime();
cout << "Execution time: " << end_time - start_time << " secs." << endl;
\end{lstlisting}
\end{frame}

\begin{frame}[fragile]
    \frametitle{Main: worker}
\begin{lstlisting}[language=c++]
bool the_end = false;
Node n;
while (!the_end && !c.has_done) {
    Message m = c.wait_for_message();
    switch(m.type) {
        case PREPARE:
            n = c.receive_node();
            dpll(n, c);
            break;
        case ALL_DONE:
            the_end = true;
            break;
    }
}
\end{lstlisting}
\end{frame}

\begin{frame}[fragile]
    \frametitle{DPLL: codice principale}
\begin{lstlisting}[language=c++]
stack<Node> s;
s.push(initial);
while (!s.empty()) {
    // Message check & (maybe) receive
    Node n = s.pop();
    if (n.contains_zero_clauses())
    {
        if (!c.is_master()) c.tell_result(SAT);
        return true;
    }
    else if (n.contains_empty_clause()) continue
    else if (/* Can apply a rule */) s.push(n);
    else // Apply splitting
}
// Ending routine
\end{lstlisting}
\end{frame}

\begin{frame}[fragile]
    \frametitle{DPLL: Sdoppiamento}
\begin{lstlisting}[language=c++]
Node n2 = n;
int a = n.find_unassigned_atom();
n.assign(a, true);
n2.assign(a, false);
s.push(n);

if (!c.is_master() || !c.give_away(n2))
    s.push(n2);
\end{lstlisting}
\end{frame}

\begin{frame}[fragile]
    \frametitle{DPLL: ricezione dei messaggi}
\begin{lstlisting}[language=c++]
if (c.any_message_waiting()) {
    Message m = c.wait_for_message();
    if (c.is_master()) {
        if (m.type == SAT) return true;
        else c.set_available(m.source);
    } else if (m.type == ALL_DONE) {
        c.has_done = true;
        return true;
    }
}
\end{lstlisting}
\end{frame}

\begin{frame}[fragile]
    \frametitle{DPLL: fase finale}
\begin{lstlisting}[language=c++]
if (!c.is_master()) {
    c.tell_result(Message::UNSAT);
    return false;
} else {
    while (c.busy_workers() > 0) {
        Message m = c.wait_for_message();
        if (m.type == SAT) return true;
        else c.set_available(m.source);
    }
    return false;
}
\end{lstlisting}
\end{frame}

\section{Prestazioni}

\begin{frame}
    \begin{itemize}
     \item File di prova: 300 clausole, 400 variabili \pause
     \item Media di 50 prove per numero di processori \pause
     \item Caveat: prestazioni DPLL dipendono molto dall'input
    \end{itemize}
\end{frame}

\begin{frame}
	\frametitle{Tempi di esecuzione}
    \begin{center}
     \pgfuseimage{grafico-tempi}
    \end{center}
\end{frame}

\begin{frame}
	\frametitle{Speed-up}
    \begin{center}
     \pgfuseimage{grafico-speed-up}
    \end{center}
\end{frame}

\begin{frame}
	\frametitle{Efficienza}
    \begin{center}
     \pgfuseimage{grafico-efficienza}
    \end{center}
\end{frame}

\part{Conclusioni}

\begin{frame}
	\partpage
\end{frame}

\begin{frame}
	\frametitle{Note positive}
	\begin{itemize}
     \item Elevata portabilità: compilatore e libreria MPI \pause
     \item Possibilità di sperimentare l'interfaccia MPI 2 per il C++
	\end{itemize}
\end{frame}

\begin{frame}
	\frametitle{Futuro}
	\begin{itemize}
	 \item Il progetto prosegue con l'esame di Fondamenti di IA \pause
	 \item Modularità: altri algoritmi per sopperire alle carenze della DPLL \pause
	 \item Perfezionamento DPLL: euristica per la scelta della variabile su cui ``sdoppiare''
	\end{itemize}
\end{frame}

\begin{frame}
    \begin{center}
        \Huge{THE END}
    \end{center}
\end{frame}

\end{document}

% That's all, folks!

